\documentclass[11pt]{article}


\begin{document}
\maketitle


\section*{Case Summary}
\textbf{Case ID:} \CaseID\\
\textbf{Objective:} Justify the Monte Carlo (MC) method used by ELMFIRE for ignition sampling by deriving the expected time-of-arrival (ToA) statistics at a target pixel when ignition is drawn uniformly from a specified mask (square or circle). We compare the empirical ToA distribution from ensembles to the mask-implied expectation, and quantify how many ensembles are needed for convergence.

\section{Assumptions}
\begin{itemize}[nosep]
  \item Flat terrain: slope $=0$, aspect arbitrary but unused for $SLP=0$.
  \item Homogeneous fuel: \texttt{IFBFM = 102}, fixed moisture: $M_1=0.03$, $M_{10}=0.04$, $M_{100}=0.05$, $M_\mathrm{lh}=0.30$, $M_\mathrm{lw}=0.60$.
  \item Steady wind: direction $\mathrm{WD20}=180^\circ$ (blowing to the north), speed $\mathrm{WS20}=15$ mph; canopy terms set to zero ($CC=CH=0$).
  \item No spotting/suppression; acceleration factor $=1$; length/width cap $\mathrm{LOW}_{\max}=8$.
  \item Coordinates are meters; the ToA along a ray from source to target uses the directional ROS returned by the ellipse kinematics.
  \item Ignition location $X$ is uniformly distributed on the mask region $\Omega$ (square or circle).
\end{itemize}


\section{Simulation Setup}
\subsection{Parameter Table}
\begin{center}
\small
\begin{tabular}{lcl}
\toprule
Fuel model & IFBFM & 102 \\
\hline
Dead fuel moisture & $M_1, M_{10}, M_{100}$ & 0.03, 0.04, 0.05 \\
Live fuel moisture & $M_\mathrm{lh}, M_\mathrm{lw}$ & 0.30, 0.60 \\
Wind (20-ft) & $(\mathrm{WD20}, \mathrm{WS20})$ & $(180^\circ, 15~\mathrm{mph})$ \\
Slope / Aspect & $(SLP, ASP)$ & $(0^\circ, 0^\circ)$ \\
Adj. factors & $(ADJ, \mathrm{ACCEL})$ & $(1, 1)$ \\
Ellipse cap & $\mathrm{LOW}_{\max}$ & 8 \\
Target (m) & $\mathbf{y}$ & $(50, 200)$ \\
Square mask (m) & $\Omega_\square$ & $[0, 100]\times[0, 100]$ \\
Circle mask (m) & $\Omega_\circ$ & center $(50,50)$, radius $R=20$ \\
\bottomrule
\end{tabular}
\end{center}

\subsection{Method (MATLAB Reference)}
For a source $\mathbf{x}\in\Omega$, we compute a direction angle $\theta=\mathrm{atan2}(y_y-x_y, y_x-x_x)$, set the fireline normal to $(\cos\theta,\sin\theta)$, update ellipse kinematics via \texttt{UX\_AND\_UY\_ELLIPTICAL}, and obtain the directional rate of spread (ROS) magnitude $v(\mathbf{x})$ (ft/min). The straight-line source-to-target distance is $d(\mathbf{x})=\lVert \mathbf{y}-\mathbf{x}\rVert$. Time of arrival is
\begin{equation}
  T(\mathbf{x})=
  \begin{cases}
    \dfrac{d(\mathbf{x})}{\,v(\mathbf{x})\cdot \frac{0.3048}{60}\,}, & \text{if coordinates in meters} \\
    \dfrac{d(\mathbf{x})}{\,v(\mathbf{x})\,}, & \text{if coordinates in feet}.
  \end{cases}
\end{equation}
The MATLAB driver (provided) evaluates $T(\mathbf{x})$ for all grid points inside the mask (uniform discrete sampling) and produces histograms and sample moments (mean/min/max).


\subsection{Input Data}
Describe input rasters, constants, initial conditions.


\subsection{Numerical Controls}
Mesh resolution, Time step(CFL), level-set solver options, etc.


\section{Expected Results and Reasoning}
Let $X\sim\mathrm{Unif}(\Omega)$ be the ignition location and $T = T(X)$ the induced ToA random variable. The \emph{mask-implied} expected value and CDF are
\begin{align}
  \mathbb{E}[T] &= \frac{1}{|\Omega|}\int_{\Omega}\! \frac{d(\mathbf{x})}{\tilde v(\mathbf{x})}\, d\mathbf{x}, 
  \quad \tilde v(\mathbf{x}) = v(\mathbf{x})\cdot \frac{0.3048}{60} \;\; (\mathrm{m/min}),\\
  F_T(t) &= \mathbb{P}\!\left(\frac{d(\mathbf{x})}{\tilde v(\mathbf{x})}\le t\right)
          = \frac{1}{|\Omega|}\int_{\{\mathbf{x}\in\Omega: d(\mathbf{x})\le t\,\tilde v(\mathbf{x})\}} d\mathbf{x}.
\end{align}
In special cases:
\begin{itemize}[nosep]
  \item If $v(\mathbf{x})\equiv v_0$ (isotropic ROS), then $T(\mathbf{x}) = d(\mathbf{x})/v_0$ and the ToA distribution is the distance distribution over $\Omega$ scaled by $1/v_0$.
  \item With an anisotropic ellipse (wind-aligned), $v(\mathbf{x})$ varies only through the direction from $\mathbf{x}$ to $\mathbf{y}$; the expectation remains an area integral, efficiently approximated by uniform quadrature on the same grid the script uses.
\end{itemize}
Thus, the MATLAB “all-points-in-mask” evaluation \emph{is} a numerical quadrature for $\mathbb{E}[T]$ and the full $F_T$, which we treat as the \emph{expected} mask-implied reference.

\section{Verification Plan with ELMFIRE Ensembles}
ELMFIRE ensembles will use identical physics and environmental settings, but ignition locations will be drawn randomly from $\Omega$ with equal probability. For an ensemble size $N$, define the empirical CDF $\hat F_T^{(N)}$ and moments (mean $\hat\mu_N$, variance $\hat\sigma_N^2$). Convergence is assessed against the mask-implied reference $(F_T, \mu, \sigma^2)$ via:
\begin{enumerate}[nosep]
  \item \textbf{Mean Relative Error (MRE):} $\mathrm{MRE} = |\hat\mu_N - \mu|/\mu$.
  \item \textbf{RMSE of histogram/bin means} over a fixed partition of $t$.
  \item \textbf{Kolmogorov–Smirnov (KS) distance:} $D_N = \sup_t |\hat F_T^{(N)}(t)-F_T(t)|$.
\end{enumerate}
A practical sample-size target for the mean uses the CLT:
\begin{equation}
  N \;\gtrsim\; \left(\frac{z_{1-\alpha/2}\,\sigma}{\varepsilon\,\mu}\right)^2,
\end{equation}
where $\varepsilon$ is the desired relative error tolerance and $z_{1-\alpha/2}$ the normal quantile. We estimate $\sigma$ from either the mask quadrature or pilot ensembles.

\section{Acceptance Criteria}
\begin{itemize}[nosep]
  \item \textbf{Mean:} $\mathrm{MRE} \le \SI{5}{\percent}$.
  \item \textbf{Distribution:} KS distance $D_N \le 0.05$ (typical); and visual agreement of PDF/CDF.
  \item \textbf{Stability:} Increasing $N$ by a factor of two changes $\hat\mu_N$ by $\le \SI{2}{\percent}$ and $D_N$ by $\le 0.01$.
\end{itemize}

\section{Results (Mask-Implied Reference)}
This section presents the direct evaluation from the MATLAB loop over all source points inside the mask:
\begin{itemize}[nosep]
  \item Histogram/PDF of ToA for the \emph{square} mask (uniform over $[0,100]^2$).
  \item Histogram/PDF of ToA for the \emph{circle} mask (center $(50,50)$, $R=20$).
  \item Empirical CDFs and summary moments (mean/min/max) for each mask.
\end{itemize}

% Auto-generated figures from your post-processing:
% Suggested filenames (update to your pipeline as needed):
%   figures/ToA_hist_square.png
%   figures/ToA_hist_circle.png
%   figures/ToA_cdf_square.png
%   figures/ToA_cdf_circle.png
% \input{figures.tex}

\section{Results (ELMFIRE Ensembles)}
After running $N$-member ELMFIRE ensembles with random ignition draws on the same mask(s):
\begin{enumerate}[nosep]
  \item Compare ensemble histograms/CDFs vs. mask-implied reference.
  \item Report $\hat\mu_N$, $\hat\sigma_N$, MRE, RMSE (bins), and KS $D_N$.
  \item Increase $N$ until criteria in \S\;5 are satisfied; record the smallest $N$ meeting tolerance.
\end{enumerate}

\section{Discussion}
\begin{itemize}[nosep]
  \item Interpret any persistent bias (e.g., due to anisotropic ellipse interacting with mask geometry).
  \item Note sensitivity to wind speed, fuel model, and target placement.
  \item Document computational cost vs.\ accuracy trade-offs for ensemble sizing.
\end{itemize}

\section*{Reproducibility}
\begin{itemize}[nosep]
  \item MATLAB script commit: \texttt{<hash>} (code in repo appendix or link).
  \item ELMFIRE build: \texttt{<compiler/flags>}, binary: \texttt{<path>}.
  \item Command(s): \texttt{./run\_case.sh}; environment: \texttt{<modules/conda env>}.
  \item Logs available under \texttt{cases/\CaseID/logs/}.
\end{itemize}


\section{Discussion}


\end{document}