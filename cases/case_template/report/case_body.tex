% Load per-case macros and prerequisites
% Shared macros for per‑case reports
\newcommand{\CaseID}{wue\_transient\_heatflux}
\newcommand{\CaseTitle}{WU--E Heat-Flux Model with Wind-Aligned Ellipse and Transient HRR}
\newcommand{\CaseOwner}{Yiren Qin}
\newcommand{\CaseVersion}{v0.1}
\newcommand{\CaseDate}{\today}

\documentclass[../report/case_report.tex]{subfiles}
\begin{document}

\subsection*{Case Summary}
\textbf{Case ID:} \CaseID\\
\textbf{Objective:} Brief objective, linking to theory/expected outcomes.


\subsection{Assumptions}
List assumptions concisely:
\begin{itemize}[nosep]
\item Homogeneous fuel; no spotting; steady uniform wind.
\item 2D structured grid, constant \(\Delta x = \Delta y\).
\item Boundary conditions: specify.
\end{itemize}


\subsection{Simulation Setup}
Key parameters and configuration choices (in elmfire.data.in). Cite the specific ELMFIRE config and any pre/post processing steps.


\subsubsection{Input Data}
Describe input rasters, constants, initial conditions.


\subsubsection{Numerical Controls}
Mesh resolution, Time step(CFL), level-set solver options, etc.


\subsection{Expected Results and Reasoning}
Derive the expected behavior (analytical solution or literature correlations). Include equations and parameter substitutions.


\subsection{Acceptance Criteria}
Define quantitative pass/fail criteria (e.g., RMSE\,\(\leq\)\,\SI{5}{\percent}).


\subsection{Results}
Discussion with figures generated in the folder `figures'. 


\subsubsection{Key Metrics}
Summarize metric values (auto‑insert from JSON if desired in a later extension) and state Pass/Fail.


\subsection{Discussion}


\end{document}
